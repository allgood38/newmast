%        File: planofattack.tex
%     Created: Mon Nov 21 12:00 PM 2011 E
% Last Change: Mon Nov 21 12:00 PM 2011 E
%
\documentclass[letterpaper]{article}
\usepackage{graphicx} % For figures and images
\usepackage{alltt} % For better verbatim

% \usepackage{fullpage}

% The HRule command is required for the title page
% from the wikibook
\newcommand{\HRule}{\rule{\linewidth}{0.5mm}}

% Change the paragraphs to block style
\setlength{\parindent}{0pt}
\setlength{\parskip}{2ex}

\begin{document}
% If the titlepage is included, uncomment below.
%%        File: /Users/allgood38/git/newmast/Documentation/title.tex
%     Created: Fri Nov 18 11:00 am 2011 E
% Last Change: Fri Nov 18 11:00 am 2011 E
%
% The root of this document is taken from the core of the Latex
% Wikibook. Not meant to be compiled on its own, instead to be
% inserted into the document using the \input command within the
% document tags.
\begin{titlepage}

% Center all the titles and junk before the minipages begin. 
\begin{center}

% Here Graphics are normally included, if there are none use
% the command directly under it to add some space, otherwise it
% doesn't look very good.
% \includegraphics[width=0.15\textwidth]{./logo}\\[1cm]    
{\ }\\[2.2cm]
\textsc{\LARGE QMAST Sailcode}\\[1.5cm]

% A sort of subtitle under the large text
\textsc{\Large Software Guide 2011}\\[0.5cm]


% Title enclosed in horizontal lines
\HRule \\[0.4cm]
{ \huge \bfseries API Documentation}\\[0.4cm]
\HRule \\[1.5cm]
% Minipages are coming up, so stop centering.
\end{center}

% The Minipage on the left-hand side
\begin{raggedright}
\begin{minipage}{0.4\textwidth}
% Insert Author/Group info here
\begin{flushleft} \large
Stephen Cripps \\
			9skc@queensu.ca \\
			06141178
			\end{flushleft}
			% End of the Author/Group info
			\end{minipage}
			\end{raggedright}

			% Space for the right-hand side minipage, note that it
			% doesn't work with long names/lines.
			%\begin{minipage}{0.4\textwidth}
			%	% Add more stuff between the flushright commands.
			%	\begin{flushright} \large
			%	\emph{Supervisor:} \\
				%	Dr.~Mark \textsc{Brown}
				%	\end{flushright}
				%\end{minipage}

				% Fills up the space between the content and the date
				\vfill

				% Bottom of the page
				\begin{center}
{\large \today}
\end{center}


\end{titlepage}




\section{Navigation with GPS Data} % (fold)
\label{sec:Navigation with GPS Data}

The boat is going to be given certain points that it needs to reach.
These will be given to it through a data file containing the GPS
locations. Given that the boat cannot sail a straight line to the
objective, it is going to need to deal with changing wind conditions and
any other obstacles.

From this point on, the locations which the robot must reach will be called
hard coded locations. In order to reach these locations, the robot will generate
a series of in between points to take care of the challenges of sailing to the
hard coded points.

% section Navigation with GPS Data (end)

\section{Sailing with Abstraction} % (fold)
\label{sec:Sailing with Abstraction}

% section Sailing with Abstraction (end)
\end{document}



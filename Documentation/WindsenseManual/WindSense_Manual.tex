%        File: WindSense_Manual.tex
%     Created: Thu Dec 01 06:00 PM 2011 EST
% Last Change: Thu Dec 01 06:00 PM 2011 EST
%
\documentclass[letterpaper]{article}
\usepackage{graphicx} % For figures and images
\usepackage{alltt} % For better verbatim
\usepackage[]{hyperref} % Links
\usepackage[]{url} % External Links
\usepackage{setspace}
\usepackage{fancyvrb}
\usepackage{color}
\usepackage[ascii]{inputenc}

%\usepackage{fullpage}

% The HRule command is required for the title page
% from the wikibook
\newcommand{\HRule}{\rule{\linewidth}{0.5mm}}

% Change the paragraphs to block style
\setlength{\parindent}{0pt}
\setlength{\parskip}{2ex}

\begin{document}
% If the titlepage is included, uncomment below.
%%        File: /Users/allgood38/git/newmast/Documentation/title.tex
%     Created: Fri Nov 18 11:00 am 2011 E
% Last Change: Fri Nov 18 11:00 am 2011 E
%
% The root of this document is taken from the core of the Latex
% Wikibook. Not meant to be compiled on its own, instead to be
% inserted into the document using the \input command within the
% document tags.
\begin{titlepage}

% Center all the titles and junk before the minipages begin. 
\begin{center}

% Here Graphics are normally included, if there are none use
% the command directly under it to add some space, otherwise it
% doesn't look very good.
% \includegraphics[width=0.15\textwidth]{./logo}\\[1cm]    
{\ }\\[2.2cm]
\textsc{\LARGE QMAST Sailcode}\\[1.5cm]

% A sort of subtitle under the large text
\textsc{\Large Software Guide 2011}\\[0.5cm]


% Title enclosed in horizontal lines
\HRule \\[0.4cm]
{ \huge \bfseries API Documentation}\\[0.4cm]
\HRule \\[1.5cm]
% Minipages are coming up, so stop centering.
\end{center}

% The Minipage on the left-hand side
\begin{raggedright}
\begin{minipage}{0.4\textwidth}
% Insert Author/Group info here
\begin{flushleft} \large
Stephen Cripps \\
			9skc@queensu.ca \\
			06141178
			\end{flushleft}
			% End of the Author/Group info
			\end{minipage}
			\end{raggedright}

			% Space for the right-hand side minipage, note that it
			% doesn't work with long names/lines.
			%\begin{minipage}{0.4\textwidth}
			%	% Add more stuff between the flushright commands.
			%	\begin{flushright} \large
			%	\emph{Supervisor:} \\
				%	Dr.~Mark \textsc{Brown}
				%	\end{flushright}
				%\end{minipage}

				% Fills up the space between the content and the date
				\vfill

				% Bottom of the page
				\begin{center}
{\large \today}
\end{center}


\end{titlepage}



\makeatletter
\def\PY@reset{\let\PY@it=\relax \let\PY@bf=\relax%
    \let\PY@ul=\relax \let\PY@tc=\relax%
    \let\PY@bc=\relax \let\PY@ff=\relax}
\def\PY@tok#1{\csname PY@tok@#1\endcsname}
\def\PY@toks#1+{\ifx\relax#1\empty\else%
    \PY@tok{#1}\expandafter\PY@toks\fi}
\def\PY@do#1{\PY@bc{\PY@tc{\PY@ul{%
    \PY@it{\PY@bf{\PY@ff{#1}}}}}}}
\def\PY#1#2{\PY@reset\PY@toks#1+\relax+\PY@do{#2}}

\def\PY@tok@gd{\def\PY@tc##1{\textcolor[rgb]{0.63,0.00,0.00}{##1}}}
\def\PY@tok@gu{\let\PY@bf=\textbf\def\PY@tc##1{\textcolor[rgb]{0.50,0.00,0.50}{##1}}}
\def\PY@tok@gt{\def\PY@tc##1{\textcolor[rgb]{0.00,0.25,0.82}{##1}}}
\def\PY@tok@gs{\let\PY@bf=\textbf}
\def\PY@tok@gr{\def\PY@tc##1{\textcolor[rgb]{1.00,0.00,0.00}{##1}}}
\def\PY@tok@cm{\let\PY@it=\textit\def\PY@tc##1{\textcolor[rgb]{0.25,0.50,0.50}{##1}}}
\def\PY@tok@vg{\def\PY@tc##1{\textcolor[rgb]{0.10,0.09,0.49}{##1}}}
\def\PY@tok@m{\def\PY@tc##1{\textcolor[rgb]{0.40,0.40,0.40}{##1}}}
\def\PY@tok@mh{\def\PY@tc##1{\textcolor[rgb]{0.40,0.40,0.40}{##1}}}
\def\PY@tok@go{\def\PY@tc##1{\textcolor[rgb]{0.50,0.50,0.50}{##1}}}
\def\PY@tok@ge{\let\PY@it=\textit}
\def\PY@tok@vc{\def\PY@tc##1{\textcolor[rgb]{0.10,0.09,0.49}{##1}}}
\def\PY@tok@il{\def\PY@tc##1{\textcolor[rgb]{0.40,0.40,0.40}{##1}}}
\def\PY@tok@cs{\let\PY@it=\textit\def\PY@tc##1{\textcolor[rgb]{0.25,0.50,0.50}{##1}}}
\def\PY@tok@cp{\def\PY@tc##1{\textcolor[rgb]{0.74,0.48,0.00}{##1}}}
\def\PY@tok@gi{\def\PY@tc##1{\textcolor[rgb]{0.00,0.63,0.00}{##1}}}
\def\PY@tok@gh{\let\PY@bf=\textbf\def\PY@tc##1{\textcolor[rgb]{0.00,0.00,0.50}{##1}}}
\def\PY@tok@ni{\let\PY@bf=\textbf\def\PY@tc##1{\textcolor[rgb]{0.60,0.60,0.60}{##1}}}
\def\PY@tok@nl{\def\PY@tc##1{\textcolor[rgb]{0.63,0.63,0.00}{##1}}}
\def\PY@tok@nn{\let\PY@bf=\textbf\def\PY@tc##1{\textcolor[rgb]{0.00,0.00,1.00}{##1}}}
\def\PY@tok@no{\def\PY@tc##1{\textcolor[rgb]{0.53,0.00,0.00}{##1}}}
\def\PY@tok@na{\def\PY@tc##1{\textcolor[rgb]{0.49,0.56,0.16}{##1}}}
\def\PY@tok@nb{\def\PY@tc##1{\textcolor[rgb]{0.00,0.50,0.00}{##1}}}
\def\PY@tok@nc{\let\PY@bf=\textbf\def\PY@tc##1{\textcolor[rgb]{0.00,0.00,1.00}{##1}}}
\def\PY@tok@nd{\def\PY@tc##1{\textcolor[rgb]{0.67,0.13,1.00}{##1}}}
\def\PY@tok@ne{\let\PY@bf=\textbf\def\PY@tc##1{\textcolor[rgb]{0.82,0.25,0.23}{##1}}}
\def\PY@tok@nf{\def\PY@tc##1{\textcolor[rgb]{0.00,0.00,1.00}{##1}}}
\def\PY@tok@si{\let\PY@bf=\textbf\def\PY@tc##1{\textcolor[rgb]{0.73,0.40,0.53}{##1}}}
\def\PY@tok@s2{\def\PY@tc##1{\textcolor[rgb]{0.73,0.13,0.13}{##1}}}
\def\PY@tok@vi{\def\PY@tc##1{\textcolor[rgb]{0.10,0.09,0.49}{##1}}}
\def\PY@tok@nt{\let\PY@bf=\textbf\def\PY@tc##1{\textcolor[rgb]{0.00,0.50,0.00}{##1}}}
\def\PY@tok@nv{\def\PY@tc##1{\textcolor[rgb]{0.10,0.09,0.49}{##1}}}
\def\PY@tok@s1{\def\PY@tc##1{\textcolor[rgb]{0.73,0.13,0.13}{##1}}}
\def\PY@tok@sh{\def\PY@tc##1{\textcolor[rgb]{0.73,0.13,0.13}{##1}}}
\def\PY@tok@sc{\def\PY@tc##1{\textcolor[rgb]{0.73,0.13,0.13}{##1}}}
\def\PY@tok@sx{\def\PY@tc##1{\textcolor[rgb]{0.00,0.50,0.00}{##1}}}
\def\PY@tok@bp{\def\PY@tc##1{\textcolor[rgb]{0.00,0.50,0.00}{##1}}}
\def\PY@tok@c1{\let\PY@it=\textit\def\PY@tc##1{\textcolor[rgb]{0.25,0.50,0.50}{##1}}}
\def\PY@tok@kc{\let\PY@bf=\textbf\def\PY@tc##1{\textcolor[rgb]{0.00,0.50,0.00}{##1}}}
\def\PY@tok@c{\let\PY@it=\textit\def\PY@tc##1{\textcolor[rgb]{0.25,0.50,0.50}{##1}}}
\def\PY@tok@mf{\def\PY@tc##1{\textcolor[rgb]{0.40,0.40,0.40}{##1}}}
\def\PY@tok@err{\def\PY@bc##1{\fcolorbox[rgb]{1.00,0.00,0.00}{1,1,1}{##1}}}
\def\PY@tok@kd{\let\PY@bf=\textbf\def\PY@tc##1{\textcolor[rgb]{0.00,0.50,0.00}{##1}}}
\def\PY@tok@ss{\def\PY@tc##1{\textcolor[rgb]{0.10,0.09,0.49}{##1}}}
\def\PY@tok@sr{\def\PY@tc##1{\textcolor[rgb]{0.73,0.40,0.53}{##1}}}
\def\PY@tok@mo{\def\PY@tc##1{\textcolor[rgb]{0.40,0.40,0.40}{##1}}}
\def\PY@tok@kn{\let\PY@bf=\textbf\def\PY@tc##1{\textcolor[rgb]{0.00,0.50,0.00}{##1}}}
\def\PY@tok@mi{\def\PY@tc##1{\textcolor[rgb]{0.40,0.40,0.40}{##1}}}
\def\PY@tok@gp{\let\PY@bf=\textbf\def\PY@tc##1{\textcolor[rgb]{0.00,0.00,0.50}{##1}}}
\def\PY@tok@o{\def\PY@tc##1{\textcolor[rgb]{0.40,0.40,0.40}{##1}}}
\def\PY@tok@kr{\let\PY@bf=\textbf\def\PY@tc##1{\textcolor[rgb]{0.00,0.50,0.00}{##1}}}
\def\PY@tok@s{\def\PY@tc##1{\textcolor[rgb]{0.73,0.13,0.13}{##1}}}
\def\PY@tok@kp{\def\PY@tc##1{\textcolor[rgb]{0.00,0.50,0.00}{##1}}}
\def\PY@tok@w{\def\PY@tc##1{\textcolor[rgb]{0.73,0.73,0.73}{##1}}}
\def\PY@tok@kt{\def\PY@tc##1{\textcolor[rgb]{0.69,0.00,0.25}{##1}}}
\def\PY@tok@ow{\let\PY@bf=\textbf\def\PY@tc##1{\textcolor[rgb]{0.67,0.13,1.00}{##1}}}
\def\PY@tok@sb{\def\PY@tc##1{\textcolor[rgb]{0.73,0.13,0.13}{##1}}}
\def\PY@tok@k{\let\PY@bf=\textbf\def\PY@tc##1{\textcolor[rgb]{0.00,0.50,0.00}{##1}}}
\def\PY@tok@se{\let\PY@bf=\textbf\def\PY@tc##1{\textcolor[rgb]{0.73,0.40,0.13}{##1}}}
\def\PY@tok@sd{\let\PY@it=\textit\def\PY@tc##1{\textcolor[rgb]{0.73,0.13,0.13}{##1}}}

\def\PYZbs{\char`\\}
\def\PYZus{\char`\_}
\def\PYZob{\char`\{}
\def\PYZcb{\char`\}}
\def\PYZca{\char`\^}
\def\PYZsh{\char`\#}
\def\PYZpc{\char`\%}
\def\PYZdl{\char`\$}
\def\PYZti{\char`\~}
% for compatibility with earlier versions
\def\PYZat{@}
\def\PYZlb{[}
\def\PYZrb{]}
\makeatother



The WindSense object is meant to emulate polling the AIRMAR wind-sensor and the compass for specific data values. Data is returned in the correct type and allows you to determine how long its been since the last update.

\section{Using the WindSense Class} % (fold)
\label{sec:Using the WindSense Class}
\begin{figure}[h]
	\HRule
\begin{Verbatim}[commandchars=\\\{\},numbers=left,firstnumber=1,stepnumber=1]
\PY{n}{Windsense} \PY{n}{airman}\PY{p}{;}
\PY{n}{Debugging} \PY{n}{panic}\PY{p}{;}

\PY{k+kt}{char} \PY{n}{debugString}\PY{p}{[}\PY{l+m+mi}{50}\PY{p}{]} \PY{o}{=} \PY{p}{\PYZob{}}\PY{l+s+sc}{'\PYZbs{}0'}\PY{p}{\PYZcb{}}\PY{p}{;}

\PY{k+kt}{void} \PY{n}{setup}\PY{p}{(}\PY{p}{)} \PY{p}{\PYZob{}}
    \PY{n}{Serial}\PY{p}{.}\PY{n}{begin}\PY{p}{(}\PY{l+m+mi}{19200}\PY{p}{)}\PY{p}{;}
    \PY{n}{airman}\PY{p}{.}\PY{n}{attach}\PY{p}{(}\PY{n}{Serial}\PY{p}{)}\PY{p}{;}

    \PY{n}{panic}\PY{p}{.}\PY{n}{attach}\PY{p}{(}\PY{n}{Serial}\PY{p}{)}\PY{p}{;}
    \PY{n}{panic}\PY{p}{.}\PY{n}{println}\PY{p}{(}\PY{l+s}{"}\PY{l+s}{Ready for action}\PY{l+s}{"}\PY{p}{)}\PY{p}{;}
\PY{p}{\PYZcb{}}

\PY{k+kt}{void} \PY{n}{loop}\PY{p}{(}\PY{p}{)} \PY{p}{\PYZob{}}
    \PY{n}{airman}\PY{p}{.}\PY{n}{pollAllValues}\PY{p}{(}\PY{p}{)}\PY{p}{;}
    \PY{n}{itoa}\PY{p}{(}\PY{n}{airman}\PY{p}{.}\PY{n}{stupidDebug}\PY{p}{(}\PY{p}{)}\PY{p}{,}\PY{n}{debugString}\PY{p}{,}\PY{l+m+mi}{10}\PY{p}{)}\PY{p}{;}
    \PY{n}{panic}\PY{p}{.}\PY{n}{println}\PY{p}{(}\PY{n}{debugString}\PY{p}{)}\PY{p}{;}
\PY{p}{\PYZcb{}}
\end{Verbatim}
	\HRule
	\caption{Example usage}
	\label{fig:ex1}
\end{figure}

Example in Figure \ref{fig:ex1} shows how to use both the debugging and windsense class in a simple program. To begin, the class needs to be instantiated at the beginning of the program. In this case the instance is called \verb:airman:. The other class instantiated is a simple debugging class for printing back to the serial port.

First note the setup loop, the Serial port has already had its baud rate set before it is passed to the airman attach function. The attach function requires that the Serial port it is given is already set-up and ready for use.

Next note that the debugging class is given the same serial port. This is okay, it just means they will be using the same port. A case like this is only good if you are plugging the arduino directly into the computer and feeding it NMEA strings for testing. 

To get information from the wind sensor, you first call a  function to get it to listen to the sensor and update its internal values. Then you can call get functions on airman to retrieve whatever information you need. By default, the pollAllValues() function will wait a maximum of two seconds for valid data to be parsed. Under normal circumstance, it will take much less time, if for some reason the function can't update all the values, it will just update the ones it can and exit with a non-zero status.

In order to retrieve the values, get functions return values from the internal \verb+structs+ within the class. Currently there are no get functions written, as they will be customized to return something meaningful

% section Using the WindSense Class (end)


\end{document}

Initialising
	Declaring a new instance

Using
	Poll for a specific data value
	Poll for all values
	Debugging
	



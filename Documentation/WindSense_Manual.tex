%        File: WindSense_Manual.tex
%     Created: Thu Dec 01 06:00 PM 2011 EST
% Last Change: Thu Dec 01 06:00 PM 2011 EST
%
\documentclass[letterpaper]{article}
\usepackage{graphicx} % For figures and images
\usepackage{alltt} % For better verbatim
\usepackage[]{hyperref} % Links
\usepackage[]{url} % External Links
\usepackage{setspace}
\usepackage{fancyvrb}
\usepackage{color}
\usepackage[ascii]{inputenc}

% \usepackage{fullpage}

% The HRule command is required for the title page
% from the wikibook
\newcommand{\HRule}{\rule{\linewidth}{0.5mm}}

% Change the paragraphs to block style
\setlength{\parindent}{0pt}
\setlength{\parskip}{2ex}

\begin{document}
% If the titlepage is included, uncomment below.
%%        File: /Users/allgood38/git/newmast/Documentation/title.tex
%     Created: Fri Nov 18 11:00 am 2011 E
% Last Change: Fri Nov 18 11:00 am 2011 E
%
% The root of this document is taken from the core of the Latex
% Wikibook. Not meant to be compiled on its own, instead to be
% inserted into the document using the \input command within the
% document tags.
\begin{titlepage}

% Center all the titles and junk before the minipages begin. 
\begin{center}

% Here Graphics are normally included, if there are none use
% the command directly under it to add some space, otherwise it
% doesn't look very good.
% \includegraphics[width=0.15\textwidth]{./logo}\\[1cm]    
{\ }\\[2.2cm]
\textsc{\LARGE QMAST Sailcode}\\[1.5cm]

% A sort of subtitle under the large text
\textsc{\Large Software Guide 2011}\\[0.5cm]


% Title enclosed in horizontal lines
\HRule \\[0.4cm]
{ \huge \bfseries API Documentation}\\[0.4cm]
\HRule \\[1.5cm]
% Minipages are coming up, so stop centering.
\end{center}

% The Minipage on the left-hand side
\begin{raggedright}
\begin{minipage}{0.4\textwidth}
% Insert Author/Group info here
\begin{flushleft} \large
Stephen Cripps \\
			9skc@queensu.ca \\
			06141178
			\end{flushleft}
			% End of the Author/Group info
			\end{minipage}
			\end{raggedright}

			% Space for the right-hand side minipage, note that it
			% doesn't work with long names/lines.
			%\begin{minipage}{0.4\textwidth}
			%	% Add more stuff between the flushright commands.
			%	\begin{flushright} \large
			%	\emph{Supervisor:} \\
				%	Dr.~Mark \textsc{Brown}
				%	\end{flushright}
				%\end{minipage}

				% Fills up the space between the content and the date
				\vfill

				% Bottom of the page
				\begin{center}
{\large \today}
\end{center}


\end{titlepage}



\input{pygment.tex}


The WindSense object is meant to emulate polling the AIRMAR wind-sensor and the compass for specific data values. Data is returned in the correct type and allows you to determine how long its been since the last update.

\section{Using the WindSense Class} % (fold)
\label{sec:Using the WindSense Class}
\begin{figure}[h]
	\HRule
\begin{Verbatim}[commandchars=\\\{\},numbers=left,firstnumber=1,stepnumber=1]
\PY{n}{Windsense} \PY{n}{airman}\PY{p}{;}
\PY{n}{Debugging} \PY{n}{panic}\PY{p}{;}

\PY{k+kt}{char} \PY{n}{debugString}\PY{p}{[}\PY{l+m+mi}{50}\PY{p}{]} \PY{o}{=} \PY{p}{\PYZob{}}\PY{l+s+sc}{'\PYZbs{}0'}\PY{p}{\PYZcb{}}\PY{p}{;}

\PY{k+kt}{void} \PY{n}{setup}\PY{p}{(}\PY{p}{)} \PY{p}{\PYZob{}}
    \PY{n}{Serial}\PY{p}{.}\PY{n}{begin}\PY{p}{(}\PY{l+m+mi}{19200}\PY{p}{)}\PY{p}{;}
    \PY{n}{airman}\PY{p}{.}\PY{n}{attach}\PY{p}{(}\PY{n}{Serial}\PY{p}{)}\PY{p}{;}

    \PY{n}{panic}\PY{p}{.}\PY{n}{attach}\PY{p}{(}\PY{n}{Serial}\PY{p}{)}\PY{p}{;}
    \PY{n}{panic}\PY{p}{.}\PY{n}{println}\PY{p}{(}\PY{l+s}{"}\PY{l+s}{Ready for action}\PY{l+s}{"}\PY{p}{)}\PY{p}{;}
\PY{p}{\PYZcb{}}

\PY{k+kt}{void} \PY{n}{loop}\PY{p}{(}\PY{p}{)} \PY{p}{\PYZob{}}
    \PY{n}{airman}\PY{p}{.}\PY{n}{pollAllValues}\PY{p}{(}\PY{p}{)}\PY{p}{;}
    \PY{n}{itoa}\PY{p}{(}\PY{n}{airman}\PY{p}{.}\PY{n}{stupidDebug}\PY{p}{(}\PY{p}{)}\PY{p}{,}\PY{n}{debugString}\PY{p}{,}\PY{l+m+mi}{10}\PY{p}{)}\PY{p}{;}
    \PY{n}{panic}\PY{p}{.}\PY{n}{println}\PY{p}{(}\PY{n}{debugString}\PY{p}{)}\PY{p}{;}
\PY{p}{\PYZcb{}}
\end{Verbatim}
	\HRule
	\caption{Example usage}
	\label{fig:ex1}
\end{figure}

Example \ref{fig:ex1} shows how to use both the debugging and windsense class in a simple program. To begin, the class needs to be instantiated at the beginning of the program. In this case 

% section Using the WindSense Class (end)


\end{document}

Initialising
	Declaring a new instance

Using
	Poll for a specific data value
	Poll for all values
	Debugging
	


